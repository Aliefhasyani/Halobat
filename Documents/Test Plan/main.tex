\documentclass[12pt,a4paper]{article}
\usepackage[utf8]{inputenc}
\usepackage{geometry}
\usepackage{fancyhdr}
\usepackage{longtable}
\usepackage{datetime}
\usepackage{array}
\usepackage{makecell}
\usepackage[hyphens]{url}
\usepackage{hyperref}
\usepackage{tabularx}

% Margin setup
\geometry{top=3cm,bottom=3cm,left=3cm,right=3cm}

% Header setup
\pagestyle{fancy}
\fancyhf{}
\fancyhead[L]{Test Plan - Halobat}
\fancyhead[R]{\today}

% Kolom fleksibel
\newcolumntype{P}[1]{>{\raggedright\arraybackslash}p{#1}}
\newcolumntype{Y}{>{\raggedright\arraybackslash}X} % untuk kolom auto wrap

\title{Rencana Pengujian Aplikasi Halobat}
\author{Kelompok 4 - Universitas Hasanuddin}
\date{\today}

\begin{document}

\maketitle
\tableofcontents
\newpage

\section{Test Plan Identifier}
Nama: Halobat Web Testing TP\_1.0

\section{References}
\begin{itemize}
    \item Proposal Proyek Perangkat Lunak: Website Katalog Obat-Obatan dengan Diagnosa dan Rekomendasi (Kelompok 4, Universitas Hasanuddin, 2025).
    \item ER Diagram Halobat dibuat menggunakan \textbf{DBML (Database Markup Language)} pada layanan \url{https://dbml.dbdiagram.io/docs}. 
    \item Definisi tabel ERD Halobat mencakup: \textit{users, roles, diagnoses, recommended\_drugs, drugs, dosage\_forms, manufacturers, active\_ingredients, brands, drug\_active\_ingredients}.
    \item Standar IEEE 829 - Test Plan Documentation.
\end{itemize}


\section{Introduction}
Pengujian ini bertujuan untuk memberikan kerangka kerja dalam memastikan kualitas aplikasi \textbf{Halobat}, yaitu website katalog obat-obatan dengan fitur diagnosa dan rekomendasi. 
Tujuan pengujian adalah:
\begin{itemize}
    \item Menjamin keakuratan informasi obat.
    \item Memastikan fitur diagnosa dan rekomendasi bekerja sesuai skenario.
    \item Memvalidasi autentikasi dan otorisasi pengguna (Admin, User).
    \item Menjamin UI/UX dapat digunakan dengan mudah oleh pengguna umum.
\end{itemize}

\section{Test Items}
\begin{itemize}
    \item Modul autentikasi (login, register, logout).
    \item Modul admin panel (CRUD data obat, pengguna, produsen).
    \item Modul katalog obat (pencarian, filter, detail obat).
    \item Modul diagnosa (input gejala, hasil diagnosa, rekomendasi obat).
\end{itemize}

\section{Features To Be Tested}
\begin{itemize}
    \item Validasi input (registrasi, login, diagnosa).
    \item Fungsi CRUD pada data obat, pengguna.
    \item Rekomendasi obat berdasarkan hasil diagnosa.
    \item Middleware autentikasi dan otorisasi.
    \item API response (status code, format JSON).
    \item Responsifitas tampilan (UI).
\end{itemize}

\section{Features Not To Be Tested}
\begin{itemize}
    \item Integrasi pihak ketiga di luar cakupan awal.
    \item Multi-bahasa.
    \item Export/print laporan.
\end{itemize}

\section{Approach}
Pengujian dilakukan dengan kombinasi:
\begin{itemize}
    \item \textbf{Unit Test}: Menggunakan PHPUnit/Pest Laravel.
    \item \textbf{Feature Test}: Menguji alur end-to-end (auth, katalog, diagnosa).
    \item \textbf{Integration Test}: Koneksi database.
    \item \textbf{Manual/Exploratory}: UI/UX dan validasi input.
    \item \textbf{Security Testing}: SQL Injection, XSS, CSRF, role-based access.
\end{itemize}

\section{Pass/Fail Criteria}
\begin{itemize}
    \item Semua test case prioritas tinggi lulus.
    \item Tidak ada bug kritis yang menghalangi fungsi utama.
    \item Target minimal 80\% unit test coverage.
\end{itemize}

\section{Suspension Criteria}
Pengujian dihentikan sementara apabila:
\begin{itemize}
    \item Sistem tidak dapat diakses (server crash).
    \item Modul autentikasi gagal berfungsi.
    \item Integritas database rusak.
\end{itemize}

\section{Test Deliverables}
\begin{itemize}
    \item Test case document.
    \item Laporan hasil eksekusi test.
    \item Laporan bug lengkap dengan langkah reproduksi.
    \item Ringkasan pengujian dan rekomendasi.
\end{itemize}

\section{Testing Tasks}
\begin{itemize}
    \item Penyusunan test plan.
    \item Dokumentasi test case.
    \item Persiapan data uji (dummy users, obat).
    \item Eksekusi test (unit, feature, integration).
    \item Laporan akhir hasil pengujian.
\end{itemize}

\section{Environmental Needs}
\begin{itemize}
    \item Laravel 11, PHP 8.2, MySQL 8.
    \item Web server: Apache/Nginx.
    \item Tools: PHPUnit/Pest, Postman, JMeter, OWASP ZAP.
    \item Data uji: dummy users (admin, user), dummy obat.
\end{itemize}

\section{Responsibilities}
\begin{itemize}
    \item \textbf{Test Manager}: koordinasi, review hasil.
    \item \textbf{Tester}: eksekusi test, dokumentasi bug.
    \item \textbf{Developer}: memperbaiki bug dan mengupdate kode.
\end{itemize}

\section{Staffing and Training Needs}
\begin{itemize}
    \item 2-3 penguji dengan pemahaman dasar software testing.
    \item Pelatihan penggunaan Laravel Testing, Postman, JMeter.
\end{itemize}

\section{Schedule}
\begin{itemize}
    \item 1-14 September 2025: Penyusunan ERD, Use Case, Activity, Sequence Diagram.
    \item 17-30 September 2025: Pembuatan API, controller, seeder, middleware.
    \item 1-7 Oktober 2025: Unit testing, API testing, bug fixing.
    \item 8-15 Oktober 2025: Retesting dan laporan akhir.
\end{itemize}

\section{Risks and Contingencies}
\begin{itemize}
    \item Risiko keterlambatan coding backend $\rightarrow$ mitigasi: paralelkan pekerjaan frontend dan backend.
    \item Risiko kurangnya pemahaman testing tools $\rightarrow$ mitigasi: training internal singkat.
    \item Risiko data dummy tidak representatif $\rightarrow$ mitigasi: gunakan data dari WHO/Depkes.
\end{itemize}

\section{Approvals}
Disetujui oleh:
\begin{itemize}
    \item Test Manager: Rezky Robbyanto Akbari
    \item Project Manager: A.M.Yusran Mazidan
\end{itemize}


\end{document}
